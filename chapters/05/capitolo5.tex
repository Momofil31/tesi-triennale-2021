\graphicspath{{./chapters/05/assets/}}
\chapter{Conclusioni}
In questo elaborato è stato presentato un resoconto di quanto ho svolto durante il periodo di tirocinio presso Cluster Reply. Posso affermare che questa esperienza è risultata molto positiva in ottica della mia formazione personale. Ho potuto infatti affrontare la mia prima esperienza lavorativa nell'ambito IT e mettere in pratica le conoscenze acquisite durante il percorso della laurea triennale. 

Cluster Reply, azienda dinamica ma ben strutturata, si è dimostrata un'ottima scelta soprattutto grazie alle persone che la compongono. I colleghi del team di lavoro, infatti, si sono rivelati molto disponibili e, nonostante le limitazioni dovute al lavoro a distanza, hanno sempre trovato il modo di aiutarmi nei momenti di difficoltà e di chiarire eventuali dubbi sui dettagli tecnici. 

\section{Tecnologie Microsoft}
Durante lo stage ho avuto la possibilità di apprendere alcune delle tecnologie del vasto catalogo di prodotti Microsoft, la cui richiesta nel mondo del lavoro è molto elevata. 

Microsoft Dynamics 365 è risultato uno strumento estremamente potente e flessibile, grazie alle ampie possibilità di customizzazione. Seppure dopo solo un mese circa di esperienza non possa considerarmi un esperto, posso affermare di aver appreso i concetti principali del suo funzionamento e le modalità secondo cui è possibile estendere le sue funzionalità, sia mediante strategie \textit{low-code/no-code}, grazie ad esempio ai Workflow e le altre possibilità del pannello di impostazioni avanzate, sia attraverso lo sviluppo di componenti software realizzati su misura come i Plugin o le Web Resource. 

Power Automate si è dimostrato uno strumento potente, grazie al grande numero di connettori disponibili, ma anche semplice e immediato nell'utilizzo, grazie alla sua interfaccia intuitiva. Inoltre, dopo aver appreso il linguaggio OData per la creazione di espressioni, è stato possibile implementare logica più complessa, aumentando così le potenzialità dello strumento.

AI Builder, infine, è risultato una tecnologia perfetta per svolgere semplici operazioni di machine learning. Non richiedendo conoscenze specifiche, il suo utilizzo risulta adatto anche ad utenti poco esperti. Tuttavia, essendo in continuo sviluppo ed estensione, le sue funzionalità possono in alcuni casi, come per la rilevazione delle firme nel secondo progetto, non essere ancora sufficienti a soddisfare le necessità dell'azienda. 


\section{Progetti sviluppati}
Riguardo ai progetti sviluppati, posso dirmi complessivamente soddisfatto, soprattutto per quanto concerne l'esito del secondo. 

Il primo progetto, mi ha permesso di apprendere i meccanismi generali di funzionamento del CRM, di Power Automate e in particolare di AI Builder. È stato molto interessante, inoltre, studiare la soluzione sviluppata dall'azienda per la gestione del processo di Business del cliente. In questo modo ho appreso come possono essere sfruttate le potenzialità di Dynamics 365 in un caso d'uso reale.
Quindi, sebbene l'obiettivo di automatizzare il processo di creazione della Planning Request non sia stato pienamente raggiunto, sono contento di aver potuto accumulare nuove competenze.

Attraverso il secondo progetto ho potuto mettere in pratica queste nuove conoscenze, applicandole ad un caso d'uso completo e ben strutturato, anche se non legato a un cliente reale. Visti i feedback positivi sul mio operato da parte del tutor aziendale, posso considerarlo un successo.

\section{Considerazioni sulla formazione universitaria}
Le conoscenze teoriche e pratiche acquisite durante il Corso di Laurea in Informatica, hanno reso possibile apprendere in poco tempo e in maniera efficace le nuove tecnologie e linguaggi utilizzati durante il tirocinio. Inoltre, mi sono sentito preparato e competente nelle attività svolte e in grado di comunicare efficacemente con i colleghi, anche in contesti in cui la mia esperienza fosse molto limitata. 
Ciò che più mi ha insegnato l'università, al di là dei linguaggi di programmazione o degli argomenti teorici, è l'approccio scientifico nei confronti dell'informatica. Ritengo, quindi, che sia quest'ultimo il fattore che consente a un laureato di poter esprimere al meglio le proprie competenze teoriche e tecniche in ambito lavorativo.