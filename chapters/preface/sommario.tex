\chapter*{Sommario} % senza numerazione
\label{sommario}

\addcontentsline{toc}{chapter}{Sommario} % da aggiungere comunque all'indiceù
Questo elaborato descrive il lavoro svolto dal laureando Momesso Filippo durante l'attività di tirocinio formativo. Lo stage ha avuto una durata di 225 ore e si è svolto presso l'azienda Cluster Reply Srl, nella sede di Silea (TV). Tuttavia, viste le restrizioni imposte dalle istituzioni a causa della pandemia da Covid-19, l'attività lavorativa si è svolta da remoto.

Durante il tirocinio, il laureando ha sviluppato due progetti finalizzati all'automazione di flussi di business utilizzando Microsoft Dynamics 365, Microsoft Power Automate e Microsoft AI Builder. Per lo sviluppo delle due soluzioni software, è stata necessaria una customizzazione a \num{360} gradi, utilizzando sia le funzionalità low-code/no-code che lo sviluppo di plug-in in linguaggio C\# e l'SDK Microsoft XRM. 

Il primo capitolo tratta il contesto aziendale, con un approfondimento sulle metodologie lavorative, la struttura dei team di lavoro e gli strumenti di Cluster Reply. Nel secondo capitolo, viene effettuata un'analisi nel dettaglio delle tecnologie Microsoft apprese e utilizzate dal tirocinante durante il periodo di stage. 
Il terzo e il quarto capitolo descrivono i progetti sviluppati mentre nel quinto viene effettuata un'analisi complessiva sull'esperienza di tirocinio. 

Lo stage formativo ha permesso al laureando di completare il percorso di laurea con un esempio reale del funzionamento del mondo del lavoro nel settore IT, oltre che apprendere nuove tecnologie che, di norma, difficilmente vengono affrontate durante il Corso di Laurea in Informatica. Pertanto l'esperienza è risultata più che positiva. 
\newpage

