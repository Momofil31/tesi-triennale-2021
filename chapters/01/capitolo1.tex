\graphicspath{{./chapters/01/assets/} }
\chapter{Contesto Aziendale}
\label{cha:intro}

\section{Il gruppo Reply}
Reply è una società specializzata in consulenze, \textit{system integration} e servizi digitali con un focus sulla concezione, design e implementazione di soluzioni basate sulle nuove tecnologie e i nuovi canali di comunicazione. Operativa dal 1996, collabora dunque con importanti realtà aziendali di diversi settori al fine di definire e sviluppare modelli di business resi possibili dai nuovi paradigmi tecnologici quali \textit{Artificial Intelligence}, \textit{Big Data}, \textit{Cloud Computing}, \textit{Digital Communication}, \textit{Internet of Things} e \textit{Social Networking}.
Il campo di azione di Reply è quello delle aziende dei settori bancario e finanziario, industriale e dei servizi, delle telecomunicazioni, dell'energia e della pubblica amministrazione.
I principali servizi offerti da Reply sono:
\begin{itemize}
  \item \textbf{Consulenze} su strategie, comunicazione, processi aziendali e tecnologie;
  \item \textbf{\textit{System Integration}} di soluzioni software esistenti;
  \item \textbf{Gestione}, monitoraggio e sviluppo continuo di sistemi e applicazioni software.
\end{itemize}

Il gruppo è formato da decine di società secondo un modello a rete e nel corso degli anni si è conquistato una posizione di prestigio nel panorama europeo e mondiale. Il fatturato 2020 si attesta a 1.250,2 milioni di Euro \cite{fatturato}.

\section{Cluster Reply}

\begin{figure}[h]
  \centering
  \includegraphics[width=0.35\textwidth]{logo-cluster-reply.png}
  \caption{Logo Cluster Reply Srl}
  \label{fig:replyClusterLogo}
\end{figure}

Cluster Reply è la società del gruppo Reply specializzata in servizi di consulenza e di integrazione di sistemi su tecnologie Microsoft. Opera in Italia in collaborazione con le altre aziende del gruppo specializzate in tecnologie Microsoft. È presente sul territorio con sedi a Milano, Padova, Roma, Torino, Trieste, Bologna e Silea. 

Per quanto riguarda la struttura, Cluster Reply è divisa in diverse sezioni, ognuna dedicata a uno specifico settore di business. Si ha una sezione dedicata a Microsoft Azure, una al sistema Microsoft ERP (\textit{Enterprise Resource Planning}), una al settore \textit{Manifacturing}, una al settore \textit{Financial} (banche e assicurazioni) e infine una sezione dedicata alla \textit{Customer Experience}.

È in quest'ultima sezione che ho svolto la mia attività di tirocinio formativo. Essa è altamente specializzata nella consulenza e realizzazione di sistemi custom basati su Microsoft Dynamics 365, la linea di applicazioni aziendali intelligenti per la pianificazione di risorse aziendali e la gestione delle relazioni con i clienti. Di questa ampia gamma di prodotti, durante il tirocinio ho avuto la possibilità di approfondire la conoscenza delle applicazioni CRM (\textit{Customer Relationship Management}) Sales Hub, Cucstomer Service Hub e un'applicazione custom specifitacamente sviluppata per un cliente.

\section{Progetti, Clienti e Prodotti}
Tra i progetti più importanti dell'azienda (non coperti da segreto professionale) si può menzionare il "sistema di automazione delle attività nella gestione degli affitti arretrati" sviluppato per il cliente Notting Hill Genesis. La soluzione proposta da Cluster Reply consiste di una soluzione CRM basata sulla piattaforma Microsoft Dynamics 365 per l'automazione  delle attività degli utenti e la comunicazione verso i clienti. Questa soluzione permette la gestione di un flusso automatizzato per la gestione degli affitti arrettrati, in grado di guidare gli utenti NHG e automatizzare le comunicazioni (SMS, email e lettere) verso i clienti, gestendone le tempistiche, i template dinamici da utilizzare e salvando l'intera documentazione su Microsoft SharePoint in caso di rinvio legale \cite{NHG}.

Un'altro prodotto confezionato e pubblicato sul Microsoft Store da Cluster Reply è il motore di configurazione di Workflow “Configurable Workflow \& SLAs Engine”. Grazie a questo prodotto software da integrare in Microsoft Dynamics 365 è possibile semplificare la configurazione e la gestione dei \textit{workflow} e degli SLA (\textit{Service Level Agreement}) \cite{configurableWorkflow}.



